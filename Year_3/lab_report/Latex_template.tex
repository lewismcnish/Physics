


\documentclass[]{article}
\usepackage{graphicx}
\usepackage{amsmath}
\usepackage{amssymb}
\usepackage{amsfonts}
\usepackage{fancyhdr}
\usepackage[headheight=65pt,tmargin=150pt,headsep=95pt]{geometry}
\usepackage{ragged2e}
\usepackage{array}
\usepackage{tabularx}



\graphicspath{{./images/}}

\pagestyle{myheadings}
\markright{Title\hfill 2663452m\hfill date\hfill}

\title{\textbf{Title}}
\author{2663452m (University of Glasgow)}
\date{date}






\begin{document}
\maketitle

\begin{abstract}
\end{abstract}
\newpage





\section*{Introduction}





\section*{Method}
To begin the experimental procedure it was first necessary to take some calibration readings based on the angle and calibration of the electromagnet.
For the angle there is some expected dependency as discussed above due to the relation $E_t = qv \times B$ where $E_t$ is the transverse electric field, $q$ is the charge of the particle, $v$ is the velocity of the particle and $B$ is the magnetic field strength.
Due to the cross product in this relationship the transverse electric field will be at a maximum when the velocity of the particle is perpendicular to the magnetic field. So by rotating the electromagnet and measuring the transverse electric field it is possible to find the angle at which the transverse electric field is at a maximum.

For the calibration of the electromagnet the voltage supplied to the magnet was varied from 0-30V and then 30-0V and a 



\section*{Analysis}




\newpage

\end{document}
